\section{Developer Documentation}

InsightCAE is a toolbox for the creation of custom workflows.
It is therefore very natural to use its API in own extensions.

There are different developer perspectives:
\begin{enumerate}
\item Creating extensions (own workflow modules).

To write own workflow modules, just a plain binary installation of InsightCAE is required. 
It brings header files for its libraries and a CMake configuration which enables the developer to build his own extensions.
It is recommended to create a custom library containing the additions (plug-in).
See section \ref{sec:plugins} for a guide.

\item Modifying and extending the InsightCAE base project itself.

Since InsightCAE is free and open software, it is perfectly possible to take it and apply extensions or corrections or any other modification or improvement.
Please note that silentdynamics GmbH develops InsightCAE primarily with its own applications in mind so there are almost certainly features missing in the API which third party users might need.
If you implement something which might be of public interest, please think about sharing it with us and the rest of the community. 
We are very open to integrate foreign contributions into the project.

Now, modifying the InsightCAE project requires to build the project from its sources. 
Please see section \ref{sec:insightcae_dev} for a guide to set up a self-built working copy of the sources.
\end{enumerate}

\subsection{Plug-In Development}
\label{sec:plugins}


\subsection{InsightCAE Development}
\label{sec:insightcae_dev}

\subsubsection{Dependencies}

The InsightCAE toolkit requires a number of dependencies. 
Not all versions have been tested, of course. 
The following list contains the major dependencies along with the versions, which are currently utilized in binary packages:
\begin{itemize}
\item boost, \url{http://www.boost.org}, 1.65.0
\item python, 3.6
\item gsl
\item armadillo
\item OpenCASCADE
\item dxflib
\item Gmsh
\item Qt, 5.
\item Wt
\item OpenFOAM
\item Code\_Aster
\end{itemize}



\subsubsection{OpenFOAM Analysis}


\begin{figure}[h!]
\centering
\digraph{openfoamanalysis}{
 rankdir=LR;
 "insight::OpenFOAMAnalysis" -> "insight::Analysis";
}
\caption{Inheritance of OpenFOAMAnalysis class}
\label{fig:}
\end{figure}

\begin{sequencediagram}
\newthread{t}{:Thread}
\newinst[1]{o}{OpenFOAMAnalysis}{:}{}
\newinst[2]{c}{OpenFOAMCase}{:}{}
\begin{call}{t}{operator()()}{o}{}
  \begin{callself}{o}{createCaseOnDisk()}{}
	\begin{callself}{o}{createCase()}{}\end{callself}
 	\begin{callself}{o}{createCaseDictsInMemory()}{}\end{callself}
 	\begin{callself}{o}{applyCustomOptions()}{}\end{callself}
 	\begin{callself}{o}{if not outputTimesPresent}{}
		\begin{call}{o}{modifyMeshOnDisk()}{c}{}\end{call}
 		\begin{callself}{o}{writeDictsToDisk()}{}
			\begin{call}{o}{createCaseOnDisk()}{c}{}\end{call}
			\begin{call}{o}{modifyCaseOnDisk()}{c}{}\end{call}
 		\end{callself}
	\begin{callself}{o}{applyCustomPreprocessing()}{}\end{callself}
 	\end{callself}
  \end{callself}
  \begin{callself}{o}{initializeSolverRun()}{}\end{callself}
  \begin{callself}{o}{runSolver()}{}\end{callself}
  \begin{callself}{o}{finalizeSolverRun()}{}\end{callself}
  \begin{callself}{o}{evaluateResults()}{}\end{callself}
\end{call}
\end{sequencediagram}
%
%\begin{itemize}
%\item 
%\end{itemize}
%1. createCaseOnDisk
%a. createCase
%b. createCaseDictsInMemory
%c. applyCustomOptions
%d. if (!outputTimesPresent)
%1. case.modifyMeshOnDisk
%2. writeDictsToDisk
%a. case.createOnDisk
%b. case.modifyCaseOnDisk
%3. applyCustomPreprocessing
%2. initializeSolverRun
%3. runSolver
%4. finalizeSolverRun
%5. evaluateResults
