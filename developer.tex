\section{Developer Documentation}

InsightCAE is a toolbox for the creation of custom workflows.
It is therefore very natural to use its API in own extensions.

There are different developer perspectives:
\begin{enumerate}
\item Creating extensions (own workflow modules).

To write own workflow modules, just a plain binary installation of InsightCAE is required. 
It brings header files for its libraries and a CMake configuration which enables the developer to build his own extensions.
It is recommended to create a custom library containing the additions (plug-in).
See section \ref{sec:plugins} for a guide.

\item Modifying and extending the InsightCAE base project itself.

Since InsightCAE is free and open software, it is perfectly possible to take it and apply extensions or corrections or any other modification or improvement.
Please note that silentdynamics GmbH develops InsightCAE primarily with its own applications in mind so there are almost certainly features missing in the API which third party users might need.
If you implement something which might be of public interest, please think about sharing it with us and the rest of the community. 
We are very open to integrate foreign contributions into the project.

Now, modifying the InsightCAE project requires to build the project from its sources. 
Please see section \ref{sec:insightcae_dev} for a guide to set up a self-built working copy of the sources.
\end{enumerate}

\subsection{Plug-In Development}
\label{sec:plugins}


\subsection{InsightCAE Development}
\label{sec:insightcae_dev}

\subsubsection{Dependencies}

The InsightCAE toolkit requires a number of dependencies. 
Not all versions have been tested, of course. 
The following list contains the major dependencies along with the versions, which are currently utilized in binary packages:
\begin{itemize}
\item boost, \url{http://www.boost.org}, 1.65.0
\item python, 3.6
\item gsl
\item armadillo
\item OpenCASCADE
\item dxflib
\item Gmsh
\item Qt, 5.
\item Wt
\item OpenFOAM
\item Code\_Aster
\end{itemize}

\subsubsection{Building InsightCAE from the Sources}

The InsightCAE sources are prepared to be built in a Posix environment.
The developers use Ubuntu Linux in the latest LTS version.
The Windows version is cross-compiled using the MXE cross compiling suite.

\subsubsection{Building in Linux}

Some prerequisites:
\begin{itemize}
\item GCC 10 should be used. Newer version were found to have buggy optimizers. \warningsymbol
\item For Ubuntu, there is a package containing all the important dependencies. 
Install it by
\begin{lstlisting}[language=bash]
$ sudo apt install insightcae-dependencies
\end{lstlisting}
\end{itemize}

\paragraph{CMake Configuration and Building}

An out-of-source build is recommended.
If the aforementioned dependencies package is used, the following variables need to be added to the CMake configuration:
\begin{itemize}
\item INSIGHT\_SUPERBUILD=/opt/insightcae
\end{itemize}

Once configured, the project is built either by
\begin{lstlisting}[language=bash]
$ make 
\end{lstlisting}
or
\begin{lstlisting}[language=bash]
$ ninja
\end{lstlisting}
depending on the chosen generator.

\paragraph{Execution}
Once built, the InsightCAE programs can be executed directly from the build folder, provided that the environment is set.

To setup the environment, add the appropriate script to the beginning of the ~/.bashrc file:
\begin{lstlisting}[language=bash]
source /path/to/build/directory/bin/insight_setenv.sh
\end{lstlisting}

\subsubsection{Building the Windows Version}

\paragraph{Cross-Compiler}
First, an installation of MXE is required.
There is Ubuntu installation package in the InsightCAE development repository (see section \ref{obtaining_insightcae}).
It contains all the required dependencies plus some additional packages (i.e. dxflib and python 3.6) and a number of patches. 
It can be installed using:

\begin{lstlisting}[language=bash]
$ sudo apt install mxe
\end{lstlisting}

On other Linux distributions, MXE probably needs to be built from its sources.
The following command should do the build of all required packages:

\begin{lstlisting}[language=bash]
$ make -j10 MXE_TARGETS=i686-w64-mingw32.shared MXE_PLUGIN_DIRS="plugins/gcc10 plugins/boost_1_66_0" pe-util cgal gettext gcc glew glfw3 mesa harfbuzz armadillo boost dxflib freetype libgcrypt glib gsl hdf5 libiconv libidn libidn2 vtk oce openssl jpeg qt5 cryptopp poppler libntlm openssl wt tiff libgsasl
\end{lstlisting}


\paragraph{Setup QtCreator IDE}

First, edit the "Kit" settings in QtCreator and create a Kit for MXE.
Therefore, add:
\begin{enumerate}
\item GCC compiler "GCC MXE32 shared", select /opt/mxe/usr/bin/i686-w64-mingw32.shared-gcc
\item C++ compiler "GCC MXE32 shared", select /opt/mxe/usr/bin/i686-w64-mingw32.shared-g++
\item Qt installation "Qt MXE32 shared", select /opt/mxe/usr/i686-w64-mingw32.shared/qt5/bin/
qmake
\item In section "CMake", add "CMake MXE shared", select /opt/mxe/usr/bin/i686-w64-mingw32.shared-cmake
\item Finally add a Kit "Desktop MXE32 shared" and select all the previously added entities
\end{enumerate}

\paragraph{Configure the Project for MXE}

Some variables need to be added to the CMake configuration:
\begin{itemize}
\item PYTHON\_INCLUDE\_DIRS=/opt/mxe/usr/i686-w64-mingw32.shared/python36/include
\item PYTHON\_LIBRARIES=/opt/mxe/usr/i686-w64-mingw32.shared/python36/python36.dll
\item VTK\_ONSCREEN\_DIR=/opt/mxe/usr/i686-w64-mingw32.shared/lib/cmake/vtk-8.2
\item CMAKE\_WRAPPER=/opt/mxe/usr/bin/i686-w64-mingw32.shared-cmake
\end{itemize}

With these variables manually added, it should be possible to build the project with QtCreator.

\paragraph{Executing the Built Binaries in a Windows Machine}

Install a SAMBA server and export the following paths as shares:
\begin{itemize}
\item the build directory (e.g. build-insight-Desktop\_MXE32\_shared-Release) as share "insightwin"
\end{itemize}
Make sure to include the option \texttt{acl allow execute always = True} for the share.
Otherwise there will be an error 0xc0000022 when it is attempted to execute an EXE file from the share.\warningsymbol

Prepare a windows machine with the following steps:
\begin{itemize}
\item install python3.6.8rc1.exe (32bit version is required!)

Make sure to enable the option "add Python executable to PATH"! \warningsymbol

\item map the share "insightwin" $\Rightarrow$ drive I:
\end{itemize}

Set the following environment variables:
\begin{itemize}
\item INSIGHT\_GLOBALSHAREDDIRS=i:\textbackslash{}share\textbackslash{}insight
\item append to PATH: 
\begin{itemize}
\item i:\textbackslash{}bin
\item i:\textbackslash{}lib
\end{itemize}
\end{itemize}

Once this is done, it should be possible to execute the EXE files directly from the network drive I:\textbackslash{}bin

\subsubsection{OpenFOAM Analysis}


\begin{figure}[h!]
\centering
\digraph{openfoamanalysis}{
 rankdir=LR;
 "insight::OpenFOAMAnalysis" -> "insight::Analysis";
}
\caption{Inheritance of OpenFOAMAnalysis class}
\label{fig:}
\end{figure}

\begin{sequencediagram}
\newthread{t}{:Thread}
\newinst[1]{o}{OpenFOAMAnalysis}{:}{}
\newinst[2]{c}{OpenFOAMCase}{:}{}
\begin{call}{t}{operator()()}{o}{}
  \begin{callself}{o}{createCaseOnDisk()}{}
	\begin{callself}{o}{createCase()}{}\end{callself}
 	\begin{callself}{o}{createCaseDictsInMemory()}{}\end{callself}
 	\begin{callself}{o}{applyCustomOptions()}{}\end{callself}
 	\begin{callself}{o}{if not outputTimesPresent}{}
		\begin{call}{o}{modifyMeshOnDisk()}{c}{}\end{call}
 		\begin{callself}{o}{writeDictsToDisk()}{}
			\begin{call}{o}{createCaseOnDisk()}{c}{}\end{call}
			\begin{call}{o}{modifyCaseOnDisk()}{c}{}\end{call}
 		\end{callself}
	\begin{callself}{o}{applyCustomPreprocessing()}{}\end{callself}
 	\end{callself}
  \end{callself}
  \begin{callself}{o}{initializeSolverRun()}{}\end{callself}
  \begin{callself}{o}{runSolver()}{}\end{callself}
  \begin{callself}{o}{finalizeSolverRun()}{}\end{callself}
  \begin{callself}{o}{evaluateResults()}{}\end{callself}
\end{call}
\end{sequencediagram}
%
%\begin{itemize}
%\item 
%\end{itemize}
%1. createCaseOnDisk
%a. createCase
%b. createCaseDictsInMemory
%c. applyCustomOptions
%d. if (!outputTimesPresent)
%1. case.modifyMeshOnDisk
%2. writeDictsToDisk
%a. case.createOnDisk
%b. case.modifyCaseOnDisk
%3. applyCustomPreprocessing
%2. initializeSolverRun
%3. runSolver
%4. finalizeSolverRun
%5. evaluateResults
